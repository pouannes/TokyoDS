\documentclass{article}
\usepackage{graphicx}
\usepackage{mathrsfs, amsmath, amsthm}
\usepackage{mathtools}
\usepackage{amssymb}
\usepackage[french, english]{babel}
\usepackage{relsize}
\usepackage{hyperref}
\usepackage{stmaryrd}
\usepackage{parskip}
\usepackage{bbold}
\usepackage{tikz}
\usetikzlibrary{arrows, arrows.meta}
\usepackage{pgfplots}

\pgfdeclarelayer{background}
\pgfsetlayers{background,main}

\pgfmathdeclarefunction{gauss}{3}{%
  \pgfmathparse{1/(#3*sqrt(2*pi))*exp(-((#1-#2)^2)/(2*#3^2))}%
}



\hypersetup{
    colorlinks=true,
    linkcolor=blue,
    filecolor=magenta,      
    urlcolor=blue,
}

\newtheoremstyle{problemstyle}                              % <name>
        {3pt}                                               % <space above>
        {3pt}                                               % <space below>
        {\normalfont}                                       % <body font>
        {}                                                  % <indent amount}
        {\bfseries}                                         % <theorem head font>
        {\normalfont\bfseries:}                             % <punctuation after theorem head>
        {.5em}                                              % <space after theorem head>
        {}                                                  % <theorem head spec (can be left empty, meaning `normal')
\theoremstyle{problemstyle}


\newtheorem{exercice}{Exercice}
\newtheorem{exemple}{Exemple}
\newtheorem{definition}{Définition}
\newtheorem{theoreme}{Théorème}
\newtheorem{remarque}{Remarque}
\newtheorem{propriete}{Propriété}



\renewcommand\thesection{\Roman{section}.}
\renewcommand\thesubsection{\quad \Alph{subsection}.}
\renewcommand\thesubsubsection{\quad \quad \alph{subsubsection}.}

\begin{document}

\title{Tokyo Data Science}
\date{{\today}}
\author{Pierre Ouannes}

\maketitle

\tableofcontents


\section{ Introduction}


Risk is expressed as 

\[
  J^*(\theta) = \sum\limits_{x} \sum\limits_{y} p_{data} (x,y) L \left( f(x, \theta), y \right)  
\]

And empirical risk:

\[
  J(\theta) = \frac{1}{m} \sum\limits_{i=1}^{m} L \left( f( x^{(i)}, \theta), y^{(i)} \right)
\]

We then expresss the gradients of risk as:

\[
  g = \nabla_{\theta} J^* ( \theta) = \sum\limits_{x} \sum\limits_{y} p_{data} (x,y) \nabla_{\theta} L \left( f(x, \theta), y \right)
\]

And that of the empirical risk:

\[
  \hat{g} = \nabla_\theta J(\theta) = \frac{1}{m} \sum\limits_{i=1}^{m} \nabla_\theta L\left( f(x^{(i)}, \theta), y^{(i)} \right)
\]

You can notice that you have the following relation between $g$ and $\hat{g}$:

\[
  \mathbb{E} [\hat{g}] = g
\]

As such, $\hat{g}$ is an unbiased estimator of g. 

\subsection{Automatic Differentiation}

\href{http://videolectures.net/deeplearning2017_johnson_automatic_differentiation}{See talk by Matthew Johnson on Automatic differentiation at the Deep Learning Summer School (DLSS) 2017}

If we have $y$ a scalar, $x$ a vector and $F$ a function that is itself made up of several functions compositions that we can write as $A$, $B$, $C$ and $D$, and the following relationship:
\[
  y = F(x) = D(C(B(A(x))))
\]

We can also construct intermediate variables:

\begin{align*}
  a &= A(x) \\
  b &= B(a) \\
  c &= C(b) \\
  y &= D(c)
\end{align*}


The following notation:
\[
  \frac{\partial a}{\partial x}
\]

refers to the derivation of vector $a$ with respect to vector $x$, and can be written as a Jacobian matrix:
\[
  \frac{\partial a}{\partial x} = 
  \begin{pmatrix}
    \frac{\partial a_2}{\partial x_1} & \frac{\partial a_1}{\partial x_2} & \dots \\
    \frac{\partial a_2}{\partial x_1} & \frac{\partial a_2}{\partial x_2} & \dots \\
    \vdots
  \end{pmatrix}
\]

Using the chain rule, we have:

\[
  \nabla_x F(x) = \frac{\partial y}{\partial x} = \frac{\partial y}{\partial c} \cdot \frac{\partial c}{\partial b} \cdot \frac{\partial b}{\partial a} \cdot \frac{\partial a}{\partial x}
\]

Each element of this chain rule is a Jacobian matrix, so this is a multiplication of matrices. 

Here, $x$ is the parameters of the Neural Network and $F(x)$ refers to the loss function. In a Neural Net, the dimensions are often reduced when we progress through the net. So $A$ reduces the dimension, $B$ also, and so on. That makes sense: we go from an high-dimension input (like an image) so a low-dimension output (a class). 


So in this chain rule, where could we start? At the end, or at the beginning? 

The answer is to begin with $ \frac{\partial d}{\partial c} $, from the left. That's because of the size of the matrices: there are much smaller than the others: we start from the loss and go backwards, and that's called \emph{back propagation}, or doing a \emph{backward pass}. Indicated in parenthesis is the way to do the computations:

\[
  \nabla_x F(x) = \frac{\partial y}{\partial x} = \left( \left(  \frac{\partial y}{\partial c} \cdot \frac{\partial c}{\partial b} \right)  \cdot \frac{\partial b}{\partial a} \right) \cdot \frac{\partial a}{\partial x}
\]

Basically, you can see it as back propagating the small matrices sizes instead of forward propagating the big matrices sizes. 

For example, let's say $  \frac{\partial y}{\partial c} $ is of size $ 10 \times 100$, $ \frac{\partial c}{\partial b} $ of size $ 100 \times 100$, $  \frac{\partial b}{\partial a} $ of size $ 100 \times 784$, and finally $ \frac{\partial a}{\partial x} $ of size $ 784 \times 784 $. 
Then the computation in a forward pass would go as :

\begin{align*}
  \left[ \frac{\partial y}{\partial x} \right] &= [10 \times 100] \cdot \left( [100 \times 100] \cdot \left( [100 \times 784] \cdot [784 \times 784] \right) \right) \\ 
  &= [10 \times 100] \cdot \left( [100 \times 100] \cdot [100 \times  784] \right) \\ 
  &= [10 \times 100] \cdot [100 \times  784] \\ 
  &= [10 \times 784] \\ 
\end{align*}

Whereas doing the computation in a backward pass would more of:

\begin{align*}
  \left[ \frac{\partial y}{\partial x} \right] &= \left( \left(  [10 \times 100] \cdot [100 \times 100] \right) \cdot [100 \times 784] \right) \cdot [784 \times 784] \\ 
  &= \left(  [10 \times 100] \cdot [100 \times 784] \right) \cdot [784 \times 784] \\ 
  &= [10 \times 784] \cdot [784 \times 784] \\ 
  &= [10 \times 784] \\ 
\end{align*}

So you see that even though you end up with the same $10 \times 784$ matrix, you can either get there by doing big matrix calculations (in a forward pass) or smaller ones (in a backward pass).

There are two main Deep Learning frameworks: PyTorch and TensorFlow. Each use different kinds of automatic differentiations engines. PyTorch use \emph{autograd}, while TensorFlow use \emph{graph based differentiation}.


Let's say you have a simple function: $x^2 + abs(x)$. Then we ask autograd: if $x=7$, what is the gradient? What autograd does is that it replaces the absolute value: $x^2 + x$, and then compute the gradient: $2x + 1$. So with $x=7$, you get $15$.

With graph based differentiation, things are a bit different. This engine does a disjunction of cases: if $x > 0$, the function is $x^2 + x$ so a gradient of $2x + 1$, wheras if $x< 0 $ the function would then be $x^2 - x$ so a gradient of $2x - 1$.

Even though you have to compute the differentiation every times with autograd, whereas you just have the path with graph based differentiation, the speed of the two engines is similar (autograd is a bit slower but not by much). On the other hand, the whole graph can be huge for complex neural networks. So as often, it's a tradeoff between storage and efficiency. Graph-based differentiation allow you to "precompile" the network and makes for faster differentiation, but at the cost of storage. 

Good exercise: cs321n's homework of designing a Neural Net in NumPy. 

Note: these days, PyTorch also uses Graph based diff, and TF also uses autograd.


\subsection{Overfitting and regressions}

\begin{figure}[h]
  \includegraphics[scale=0.7]{fittings}
  \caption{Underfitting vs overfitting}
\end{figure}

How to avoid overfitting? For example, let's say you have a polynomial:

\[
  P(x) = a_0 + a_1x + a_2 x^2+ \dots + a_{20}x^{20}
\]

You can try to add something to the loss fonction that constrains the size of the coefficient, something like:

\[
  R(\textbf{a}) = a_0^2 + a_1^2 + \dots + a_{20} ^2 = \sum\limits_{i=0}^{20} a_i ^ 2
\]

Then we can add $R$ to the loss to try to avoid overfitting:

\[
  Loss + \alpha \underbrace{\sum\limits_{i=0}^{20} a_i ^ 2}_{\text{Regularization} }
\]

$\alpha$ is an hyperparameter that regulates how much regularization we want to add.

Is it possible to overfit while doing linear regression? Let's say you have:

\[
  y = w_0 + w_1 x_1
\]

Then you probably won't overfit. However, if you have a linear regression with a lot of variables:

\[
  y = w_0 + \sum\limits_{i=1}^{2000} w_i x_i
\]

Then you may overfit. One example of that may be a survey. 

\subsubsection{Multivariate regression}

Multivariate regression is something like:

\[
  \hat{y} (w,x) = w_0 + w_1x_1 + \dots + w_p x_p
\]

$x_i$ are called \emph{regressors} in statistics, and \emph{features} in Machine Learning. In a neural networks, $w_i$ for $i > 0$ would be called the \emph{weights}, and $w_0$ the \emph{bias} and is rather written as $b$.

To make the notation simpler, we'll rather write:

\[
  \hat{y} (w,x) = w_0x_0 + w_1x_1 + \dots + w_p x_p
\]

With $x_0 =1$. It's exactly the same thing, but it's not simpler to write for example with a single sum:

\[
  \hat{y} (w,x) = \sum\limits_{i=0}^{p} w_i x_i \quad \text{instead of}  \quad \hat{y} (w,x) = w_0 +  \sum\limits_{i=1}^{p} w_i x_i
\]

We can write the loss as:

\[
  Loss = \frac{1}{m} \sum\limits_{i=1}^{m} \left( y^{(i)} - \sum\limits_{j=1}^{p} w_j x_j \right) ^{2}
\]

We can also right it as:

\[
  Loss = \frac{1}{m} \sum\limits_{i=1}^{m} \left( y^{(i)} - \textbf{w}\cdot \textbf{x} \right) ^{2}
\]

With $\textbf{x} = (x_1, \dots , x_p)$ and $\textbf{w} = (w_1, \dots , w_p)$.

Finally, we can also right is as:

\[
  Loss = \frac{1}{m} \left\lVert X \cdot \textbf{w} - \textbf{y} \right\rVert_2 ^{2}
\]

In this expression, $\textbf{w}$ is the vector from before but now in a column, to make the matrix multiplication work, while $X$ is:

\[
  X =
  \begin{bmatrix}
    x_1^{(1)} & x_2^{(1)} & \dots & x_p^{(1)} \\
    x_1^{(2)} & x_2^{(2)} & \dots & x_p^{(2)} \\
    \vdots    & \vdots    & \ddots& \vdots    \\
    x_1^{(m)} & x_2^{(m)} & \dots & x_p^{(m)}
  \end{bmatrix}
\]

So it's a $m \times p$ matrix, with each column containing a given observation ($m$ is the number of observations, and $p$ the number of parameters for a given observation). 

What is $ \left\lVert \cdot \right\rVert _2 $? It's the $L^2$ norm, defined as, for a vector $v$:

\[
  \left\lVert v \right\rVert_2 = (v_1 ^2 + v_2^2 + \dots v_p^2) ^{1/2}
\]

More generally, $L^n$ norm is defined as:


\[
  \left\lVert v \right\rVert_n = (v_1 ^n + v_2^n + \dots v_p^n) ^{1/n}
\]

So now we have the loss fully vectorized, as an $L^2$ norm, a matrix multiply and a difference of vectors. To be clear, our predictions is $\hat{\textbf{y}} = X \cdot \textbf{w}$, so 
\[
  X \cdot \textbf{w} - \textbf{y} = \hat{\textbf{y}}- \textbf{y}
\]

is just the difference between the prediction and the ground truth. 

This notation is also the one used in scikit-learn. \href{https://scikit-learn.org/stable/modules/linear_model.html#ordinary-least-squares}{See the documentation for linear regression here for example.}

\subsubsection{Overfitting and hypothesis testing}

So back to the overfitting issue. How to know if you're overfitting?
One way is to split the dataset into the training set and the validation set. You train on the training set, and check how the model is doing on the validation set. If the model is doing very good on the training set but not the validation set, it means the model is overfitting on the training set and regularization is needed. 

How to know wether to include a particular variable or not? You do hypothesis testing, a technique common in statistics. We'll cover that later on. These days, using a validation test is mostly used because of the quantity of data we have. 

\subsubsection{Ridge Regression ($L^2$ regularization)}

Note: in functional analysis, if:

\[
  \int \left\lvert f(x) \right\rvert ^{2} < \infty
\]

Then we say that $f$ is in $L^2$ space.

So, back to we what we had: we want to minimize the following loss:

\[
  Loss = \frac{1}{m} \left\lVert X \cdot \textbf{w} - \textbf{y} \right\rVert_2 ^{2}
\]

And we can add a regularization loss to that:

\[
  Loss = \frac{1}{m} \left\lVert X \cdot \textbf{w} - \textbf{y} \right\rVert_2 ^{2} + \alpha \left\lVert \textbf{w} \right\rVert_2 ^{2}
\]

In neural networks, the $\alpha \left\lVert \textbf{w} \right\rVert_2 ^{2}$ term is called \emph{weight decay}.

In a large model, there might be a lot of parameters and we might want a sparser representation: with a lot of parameters set to $0$. There is a kind of regression that does that: \emph{Lasso regression}.

\subsubsection{Lasso regression ($L^1$ regularization)}

Let's take the $L^1$ norm instead of $L^2$:

\[
  Loss = \frac{1}{m} \left\lVert X \cdot \textbf{w} - \textbf{y} \right\rVert_2 ^{2} + \alpha \left\lVert \textbf{w} \right\rVert_1
\]

As a reminder, $L^1$ regularization is:

\[
  \left\lVert \textbf{w} \right\rVert_1 = \left\lvert w_1 \right\rvert + \left\lvert w_2 \right\rvert + \dots + \left\lvert w_p \right\rvert 
\]

As an example, let's say we have:
\begin{itemize}
  \item $w_1 = 3.1$ associated with $x_1$ mesures how many hours spent studying microeconomocs
  \item $w_2 = 0$ associated with $x_2$ mesures how many hours spent studying macroeconomics
  \item $w_3 = 5$ associated with $x_3$
  \item $w_4 = 0$ associated with $x_4$
  \item $y$ is how well you do in the macroeconomics exam.
\end{itemize}

However, Lasso can set $w_2$ to $0$ even though it doesn't make intuitive sense here: the number of hours spent studying macroeconomics should be the most important variable predicting the macroeconomics test score. 

What can happens is that $x_1$ and $x_2$ can be highly correlated: people that study a lot study a lot for both micro and macro, and people that don't study a lot don't study a lot for both micro and macro. So there's two features that are really the same, and the Lasso regression doesn't really know which one to pick and might pick to wrong one based on nothing else than noise. 

To say that another way: a parameter equals to 0 in Lasso regression might indeed not be relevant, but also it might be the case that it's just highly correlated with another parameter and Lasso choose the other one. This phenomenon is also sometimes called \emph{multicollinerarity}.

So Lasso regression is great for getting parsimonious predictions but not that useful to learn about causal relationships. 

\subsubsection{Elastic Net ($L^1$ and $L^2$ regularization)}

(nothing to do with Neural Nets)

This time, we use both $L^1$ loss \emph{and} $L^2$ loss: 

\[
  Loss = \frac{1}{m} \left\lVert X \cdot \textbf{w} - \textbf{y} \right\rVert_2 ^{2} + \alpha_1 \left\lVert \textbf{w} \right\rVert_1 + \alpha_2 \left\lVert \textbf{w} \right\rVert_2 ^{2}
\]




\end{document}